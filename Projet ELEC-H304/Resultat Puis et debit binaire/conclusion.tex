\section{Conclusion}

Ce projet a démontré l'application pratique de concepts théoriques par la création d'un simulateur de ray-tracing. En adoptant Python et une méthodologie de modélisation mathématique par la méthode des images, un outil capable d’évaluer la propagation des ondes et leurs interactions avec les matériaux dans un environnement complexe a été développé.

L'utilisation de l'architecture multicœur a permis de réduire significativement les temps de calcul, notamment lors de la génération des heatmaps, optimisant ainsi l'évaluation de la couverture de l'appartement. De plus, l'optimisation de l'emplacement des émetteurs a amélioré la couverture du signal dans l'ensemble de l'appartement.

Les résultats confirment que le simulateur peut calculer avec précision les variations de puissance et de débit binaire dans divers scénarios au sein d'un appartement modélisé. La structure du code, orientée objet, a facilité la gestion et l'évolution du logiciel, offrant la modularité et l'extensibilité nécessaires pour l'intégration future de modèles plus complexes.
