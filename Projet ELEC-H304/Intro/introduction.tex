Dans le cadre du cours de Physique des Télécommunications donné par Monsieur De Doncker à l'Université Libre de Bruxelles, un projet de simulation de ray-tracing a été demandé. L'objectif de ce projet est de déterminer la puissance reçue et le débit binaire en chaque point d'un appartement simulé.

Pour commencer, les hypothèses seront abordées et la modélisation physique mise en place pour faciliter la réalisation de la simulation. Cette étape permet de définir les paramètres et les conditions du modèle, permettant ainsi d'obtenir des résultats cohérents et représentatifs.

Ensuite, la partie simulation sera expliquée plus en détail, en justifiant le choix du langage informatique et en détaillant le squelette du code. Il est essentiel de choisir un langage adapté qui offre à la fois des performances efficaces et une facilité de mise en œuvre des algorithmes de ray-tracing.

Par la suite, le code sera validé. Pour ce faire, l'exercice 4.1 du syllabus du cours sera pris comme exemple. Les résultats de la simulation seront comparés avec ceux obtenus lors des séances d'exercice, afin de vérifier la précision et la fiabilité de mon implémentation.

Enfin, le code sera appliqué à un appartement représentant un exemple plus complexe. Les optimisations du code pour le temps de calcul et de la couverture optimale de l'appartement par un et deux émetteurs seront expliquées. Les différents résultats obtenus seront affichés selon les paramètres définis dans la description du projet.

Ce projet vise à fournir une description détaillée des résultats obtenus grâce au logiciel de ray-tracing. Il offre aux étudiants l'opportunité de se familiariser avec les concepts théoriques abordés en cours et de les appliquer directement à un problème concret dans le domaine de l'ingénierie des télécommunications.
