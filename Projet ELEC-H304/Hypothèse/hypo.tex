Le cadre de ce projet repose sur plusieurs hypothèses clés qui sont détaillées ci-dessous :
\begin{enumerate}
    \item L'étude est limitée à une configuration bidimensionnelle où l'émetteur et le récepteur sont placés à la même hauteur du sol.
    \item Les ondes sont considérées comme se propageant uniquement dans le plan horizontal, avec une polarisation perpendiculaire à ce plan.
    \item La diffraction par les obstacles n'est pas prise en compte dans ce modèle.
    \item La puissance moyenne est calculée au centre d'un carré fictif de côté \(0.5\) mètres.
    \item Les antennes sont modélisées comme des dipôles idéaux émettant une puissance de \(20\) dBm.
    \item L'épaisseur des murs est négligée dans le tracé des rayons.
\end{enumerate}
Ces simplifications facilitent le calcul des grandeurs physiques importantes et la compréhension des principes de base de la propagation des ondes.

\section{Modélisation Mathématique}
La hauteur équivalente des antennes, dans le contexte de la seconde hypothèse, est simplifiée par l'équation suivante :
\begin{equation}
    h_e = -\frac{\lambda}{\pi}
    \label{batard}
\end{equation}
où \( \lambda \) est la longueur d'onde calculée par \( \lambda = \frac{c}{f} \), avec \( c \) la vitesse de la lumière et \( f \) la fréquence de l'émetteur (\(60\) GHz). En substituant ces valeurs dans \ref{batard} :
\[
    h_e = -\frac{299792458}{60 \times 10^9 \times \pi} \approx -1.59045 \times 10^{-3} \text{ mètres}
\]

Cette hauteur constante simplifie l'expression de la puissance reçue, utilisant l'équation du cours :
\begin{equation}
P_{RX} = \frac{1}{8 R_a} \left|\sum_{n=1}^{N} \vec{h}_e^{RX}\left(\frac{\pi}{2}, \phi_n\right) \underline{E}_n(\vec{r})\right|^2
\end{equation}

Le gain de l'antenne émettrice est déterminé par :
\begin{equation}
    G_{TX} = \frac{\pi Z_0}{Ra}\frac{|\vec{h_{e\perp}}|^2}{\lambda^2} \approx 1.64
\end{equation}


\section{Caractéristiques des Matériaux}
Les propriétés électromagnétiques des matériaux sont indispensables pour les calculs de réflexions et transmissions qui diffèrent en fonction des propriétés du matériau.

\begin{table}[h]
\centering
\begin{tabular}{|l|c|c|}
\hline
\textbf{Matériau} & \textbf{Permittivité relative} & \textbf{Conductivité (S/m)} \\ \hline
Brique           & 3.95                           & 0.073                      \\ \hline
Béton            & 6.4954                         & 1.43                       \\ \hline
Cloison          & 2.7                            & 0.05346                    \\ \hline
Vitre            & 6.3919                         & 0.00107                    \\ \hline
Paroi métallique & 1                              & $10^7$                        \\ \hline
\end{tabular}
\caption{Propriétés électromagnétiques des matériaux}
\label{tab:material-properties}
\end{table}

\section{Sensibilité du Récepteur et Débit Binaire}
La sensibilité du récepteur est corrélée de manière linéaire au débit binaire maximal sur une échelle logarithmique, comme indiqué dans le tableau suivant :
\begin{table}[h]
\centering
\begin{tabular}{|c|c|}
\hline
\textbf{Sensibilité (dBm)} & \textbf{Débit binaire maximal} \\
\hline
-90 & 50 Mb/s \\
-40 & 40 Gb/s \\
\hline
\end{tabular}
\caption{Corrélation entre la sensibilité et le débit binaire}
\label{tab:sensitivity}
\end{table}

Au-delà des extrêmes de cette corrélation, la communication est soit impossible (pour une sensibilité inférieure à \(-90\) dBm) soit plafonnée (pour une sensibilité supérieure à \(-40\) dBm). En présence de plusieurs émetteurs, le récepteur choisira celui qui offre la plus grande puissance de signal, sans sommer les puissances des autres émetteurs.
